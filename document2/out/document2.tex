\documentclass{article}
% generated by Madoko, version 1.1.3
%mdk-data-line={1}


\usepackage[heading-base={2},section-num={False},bib-label={hide},fontspec={True}]{madoko2}
\usepackage[UTF8]{ctex}


\begin{document}



%mdk-data-line={3}
\section{\mdline{3}1.\hspace*{0.5em}\mdline{3}Elasticsearch性能基准测试}\label{sec-elasticsearch}%mdk%mdk

%mdk-data-line={5}
\subsection{\mdline{5}1.1.\hspace*{0.5em}\mdline{5}官方推荐机器配置}\label{section}%mdk%mdk

%mdk-data-line={6}
\begin{itemize}[noitemsep,topsep=\mdcompacttopsep]%mdk

%mdk-data-line={6}
\item\mdline{6}内存    \mdline{6}\mdbr
\mdline{7}es最重要的资源是内存,64GB RAM的机器是官方推荐的,32、16GB的机器也能用于生产环境,但不建议小于8GB。%mdk

%mdk-data-line={8}
\item\mdline{8}CPU    \mdline{8}\mdbr
\mdline{9}es对CPU开销不是很大(本机开了3个node,同时进行100并发查询、数据插入索引,才把cpu将将打满100\%),但多核CPU还是必须的,通用的集群使用2到8核机器。%mdk

%mdk-data-line={10}
\item\mdline{10}硬盘    \mdline{10}\mdbr
\mdline{11}任何集群的瓶颈,对于磁盘io要求很高(如:大量、实时写入,大量存取索引),最好能使用SSD。%mdk

%mdk-data-line={12}
\item\mdline{12}网络    \mdline{12}\mdbr
\mdline{13}由于elasticsearch的特性,集群节点间的传输,包括数据的分片、备份、同步,以及数据从不同节点的分片获取和汇总。网络带宽也是很重要的。%mdk
%mdk
\end{itemize}%mdk

%mdk-data-line={16}
\subsection{\mdline{16}1.2.\hspace*{0.5em}\mdline{16}性能试验方案}\label{section}%mdk%mdk

%mdk-data-line={17}
\subsubsection{\mdline{17}1.2.1.\hspace*{0.5em}\mdline{17}数据库导入}\label{section}%mdk%mdk

%mdk-data-line={18}
\noindent\mdline{18}采用logstash导入,以及kettle导入,并比较两者效率。主要测试磁盘io、网络。%mdk

%mdk-data-line={20}
\subsubsection{\mdline{20}1.2.2.\hspace*{0.5em}\mdline{20}本地json导入}\label{sec-json}%mdk%mdk

%mdk-data-line={21}
\noindent\mdline{21}通过本地json文件导入。主要测试index速度和效率。%mdk

%mdk-data-line={23}
\subsubsection{\mdline{23}1.2.3.\hspace*{0.5em}\mdline{23}内存}\label{section}%mdk%mdk

%mdk-data-line={24}
\noindent\mdline{24}1、利用jstat日志来得到内存指标的变化,\mdline{24}\mdcode{jstat~-gc~-h5~XXX~3s~\textgreater{}~test-100W.log}\mdline{24},不同数据量修改文件名,输出到不同文件中。    \mdline{24}\mdbr
\mdline{25}2、粗粒度的监控Java Heap、GC,采用Grafana监控模式查看,通过http resuful api调用(查询)或siege压测的方式查看操作时的内存指标变化。    \mdline{25}\mdbr
\mdline{26}3、也可通过Visualvm的Heap dump查看相关指标。%mdk

%mdk-data-line={28}
\subsubsection{\mdline{28}1.2.4.\hspace*{0.5em}\mdline{28}查询效率}\label{section}%mdk%mdk

%mdk-data-line={29}
\noindent\mdline{29}通过siege、multiprocess检测查询平均耗时。以及测试优化点。%mdk

%mdk-data-line={31}
\subsubsection{\mdline{31}1.2.5.\hspace*{0.5em}\mdline{31}官方的benchmark指标检测}\label{sec-benchmark}%mdk%mdk

%mdk-data-line={32}
\noindent\mdline{32}配置esrally环境。https://github.com/elastic/rally%mdk

%mdk-data-line={34}
\subsection{\mdline{34}1.3.\hspace*{0.5em}\mdline{34}数据准备}\label{section}%mdk%mdk

%mdk-data-line={35}
\noindent\mdline{35}已准备mysql的数据表,数据规模130万。可扩展。
已准备与上面相同的json文件,数据规模130万。可扩展。%mdk

%mdk-data-line={38}
\subsection{\mdline{38}1.4.\hspace*{0.5em}\mdline{38}相关脚本}\label{section}%mdk%mdk

%mdk-data-line={39}
\noindent\mdline{39}例如下面,为es创建index脚本,其他脚本见文件夹下:%mdk
\begin{mdpre}%mdk
\noindent curl~-XPUT~"http://127.0.0.1:9200/productindex"\\
curl~-XPOST~"http://127.0.0.1:9200/productindex/product/\_mapping?pretty"~-d~'~\\
\{\\
\preindent{4}"product":~\{\\
\preindent{12}"properties":~\{\\
\preindent{16}"company\_name":~\{\\
\preindent{20}"type":~"text",\\
\preindent{30}"analyzer":~"ik\_smart",\\
\preindent{20}"search\_analyzer":~"ik\_smart"\\
\preindent{16}\},\\
\preindent{16}"id":~\{\\
\preindent{20}"type":~"long"\\
\preindent{16}\},\\
\preindent{16}"data\_status":~\{\\
\preindent{20}"type":~"long"\\
\preindent{16}\}\\
\preindent{12}\}\\
\preindent{8}\}\\
\preindent{2}\}'\\
\preindent{2}%mdk
\end{mdpre}\noindent\mdline{63}create\mdline{63}\emph{index}\mdline{63}*\mdline{63}.sh: 创建index以及index配置。        \mdline{63}\mdbr
\mdline{64}*\mdline{64}logstash.conf: logstash配置从mysql同步至elasticsearch    \mdline{64}\mdbr
\mdline{65}multi\mdline{65}\_\mdline{65}search.py: 多进程压测    \mdline{65}\mdbr
\mdline{66}elasticsearch\mdline{66}\_\mdline{66}metrics.py: 循环获取系统指标供grafana监控。  \mdline{66} \mdline{66} 
\begin{mdpre}%mdk
\noindent\preindent{18}%mdk
\end{mdpre}\begin{mdflushright}%mdk
\noindent\mdline{69}\hspace*{1ex}\hspace*{1ex}\mdline{69}
Enjoy! By Hanbing.    \mdline{70}\mdbr
\mdline{71}Using madoko.%mdk
\end{mdflushright}%mdk


\end{document}
